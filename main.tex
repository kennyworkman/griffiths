\documentclass[10pt]{article}
\usepackage{kennyworkman}

% https://www.reed.edu/physics/faculty/griffiths/
\usepackage{graphicx}
\def\rcurs{{\mbox{$\resizebox{.16in}{.08in}{\includegraphics{ScriptR}}$}}}
\def\brcurs{{\mbox{$\resizebox{.16in}{.08in}{\includegraphics{BoldR}}$}}}
\def\hrcurs{{\mbox{$\hat \brcurs$}}}

\newcommand{\uvec}[1]{\boldsymbol{\hat{\textbf{#1}}}}

\title{Griffiths: Introduction to Electrodynamics}
\author{Kenny Workman}
\date{\today}

\begin{document}

\maketitle

\section{Mathematical Foundations}

\section{Electrostatics}

In electrodynamics, we are often interested in the force a collection of source charges $q_1 \dots q_n$ exerts on a test charge $Q$.

This problem becomes difficult if our source charge(s) are moving, so we constrain our cases to those where these charges are fixed. Electrostatics is the study of these cases.

\subsection{The Electric Field}

\begin{definition}
	\textbf{The principle of superposition} tells us the force exerted from
	different charges are independent from each other. If $F_1$ is the force of
	$q_1$ acting on $Q$, then the force from the collection $\{q_i\}$ is simply $F = \sum_i q_i$.
\end{definition}

Some notational convention: 
\begin{itemize}
	\item{$r'$ is a position vector of a source charge}
	\item{$r$ is a position vector of a test charge}
	\item{$\rcurs = r' - r$, the distance vector between our source and test}
\end{itemize}

\begin{definition}
	\textbf{Coulomb's law} gives us the force on our test charge $Q$ from a
	single source charge $q$:
	\[F = \frac 1 {4\pi\epsilon_0} \frac {qQ} {\rcurs^2} \hrcurs \]
\end{definition}

\begin{definition}
	We can start to think of the force from $\{q_i\}$ in aggregate by factoring a common $Q$ from a sum of forces:
	\[F = Q\boldsymbol{E}\]
	\[\boldsymbol{E} = \frac 1 {4\pi\epsilon_0} \sum_i \frac{q_i}{\rcurs^2} \hrcurs\]
	$\boldsymbol{E}$ is called the \textbf{electric field} of the source charges $\{q_i\}$.
\end{definition}

Notice that $E$ is a vector function dependent on the location of our test
charge. This function is a mathematical description of our "field" and we are
encouraged to think of it as some jelly-like substances that permeates space.

This gives machinery if we have a discrete collection of points. However, often
our charge is more like a continuous schmear over some line, surface or volume,
motivating new expansions of Coulomb's law:

\begin{definition}[Coulomb's for continuous line charge]
	$E = \frac 1 {4\pi\epsilon_0} \int_l \frac{\lambda(r')}{\rcurs^2} \hrcurs \dd \ell$
\end{definition}

Here we are integrating over the contribution of each infinitessimal section of
the continuous charge to the field. Notice that multiple things are "moving" - the
displacement vector $\rcurs$ as well as the source charge (if it is not
constant) $\lambda(r')$. 

This extends to surfaces and volumes:

\begin{definition}[Coulomb's for continuous surface charge]
	$E = \frac 1 {4\pi\epsilon_0} \int \frac{\sigma(r')}{\rcurs^2} \hrcurs \dd a$
\end{definition}

\begin{definition}[Coulomb's for continuous volume charge]
	$E = \frac 1 {4\pi\epsilon_0} \int \frac{\rho(r')}{\rcurs^2} \hrcurs \dd \tau$
\end{definition}

\begin{example}{2.2}
Calculation for the field distance $z$ from the midpoint of a line of length $2L$ with constant charge $\lambda$.

We use Coulomb's law:
	\[
		\int_{-L}^{L} \frac 1{4\pi\epsilon_o} \frac{\lambda}{\rcurs^2} \hrcurs \dd \ell
	\]
where:
\[ \rcurs^2 = z^2 + x^2\]
\[ \hrcurs = \frac{x\uvec{x} + z\uvec{z}}{\sqrt{z^2 + x^2}} \]
\end{example}
\indent Skipping integration steps we are left with our field $\boldsymbol{E}$. Note our field is in general a function of our field point but what we provide here is an evaluation of this function.

\[
	\boldsymbol{E} = \frac 1{4\pi\epsilon_o} \frac{2\lambda L}{z \sqrt{z^2 + L^2}} \uvec{z}
\]

\begin{exercise}[2.4]
	Calculate the electric field distance $z$ away from the midpoint of a square loop of constant charge $\lambda$ and side $a$.
\end{exercise}

\begin{solution}[2.4]

The field is symmetric with respect to each side of the square, so we can compute a value for each side and add them up using the principal of superposition.

To find the $\boldsymbol{E}'$ contributing from a single side, proceed as in $(2.2)$, but set $z = \sqrt{z^2 + (\frac a2)^2}$. We are interested in the vertical component of this vector, as horizontal components will cancel from opposing sides of the square, so our result is $4\cos(\theta)\boldsymbol{E}'$ where $\cos(\theta) = \frac{z}{\sqrt{z^2 + (\frac a2)^2}}$ and $4$ comes from each side.

\end{solution}

\begin{exercise}[2.7]
	Calculate the electric field distance $z$ from the surface of a sphere
	centered at the origin with radius $R$ and constant charge $\lambda$.
\end{exercise}

\begin{solution}
	\[ \bold{r} =  z\uvec{z}\]
	\[ \bold{r'} = R(\sin\theta\cos\phi\uvec{x} + \sin\theta\sin\phi\uvec{y} + \cos\theta\uvec{z}) \]
	\[ \brcurs = \bold{r} - \bold{r'} \]
	\[ E = \frac 1 {4\pi\epsilon_0} \int_\theta \int_\phi \frac \lambda {\rcurs^2} \hrcurs \dd a\]

	Really do not want to calculate this and will use as motivation for using
	Gauss's law.

\end{solution}

\section{Divergence and curl of electric fields}


\section{Appendix}

\end{document}
